\chapter{Fundamentos Matemáticos de la Teoría de la Computación}

Muchos problemas en fundamentos de computación pueden verse como problemas concernientes a conjuntos.

\section{Conjuntos}

Un conjunto es una colección de objetos distinguibles.
\textbf{Notación: }$b\in L$ $x\not\in L$.

En un conjunto no se repiten los elementos, el orden es irrelevante.

\subsection{Conjunto Unitario}
Es aquel formado por un único elemento.
$$U=\lbrace 1 \rbrace$$
\subsection{Conjunto Vacío}
Es el conjunto sin elementos.
\textbf{Notación: }$\phi$ o $\lbrace \rbrace$.
\subsection{Conjuntos Importantes}
\begin{itemize}
\item Enteros no Negativos $\mathds{N}$
\item Enteros Positivos $\mathds{N}^{+}$
\item Enteros $\mathds{Z}$
\item Racionales $\mathds{Q}$
\item Reales $\mathds{R}$
\item Complejos $\mathds{C}$
\end{itemize}

Podemos expresar algunos conjuntos mediante referencia a otros conjuntos y a las propiedades que los elementos pueden o no tener.

Si $F=\lbrace 1,3,9 \rbrace \quad G=\lbrace 3,9\rbrace \\ G=\lbrace x/ x \in F \land x>2\rbrace \\ B=\lbrace x/x \in A \land x \mbox{ tiene la propiedad P}\rbrace$

Podemos expresar el conjunto de los números naturales impares.
$$I=\lbrace x/x \in \mathds{N} \land \mbox{x no es divisible por 2}\rbrace$$

\subsection{Subconjuntos}
A es un subconjunto de B si cada elemento de A también está en B. \textbf{Notación: }$A\subseteq B$.
Ejemplo: $I\subseteq \mathds{N} \quad A\subseteq A$

\subsection{Subconjunto Propio}
Si A es subconjunto de B pero es diferente a B se le llamará subconjunto propio de B. \textbf{Notación: }$A \subset B$.

\subsection{Conjunto Universal}
También conocido como Universo del Discurso. \textbf{Notación: }$\mathfrak{U}$.
\subsection{Complemento de un Conjunto}
Dado un conjunto $A\subseteq \mathfrak{U}$ el conjunto de A se denota por $\bar{A}$ y está definido por.
$$\bar{A}=\lbrace x/x \in \mathfrak{U} \land x\in A\rbrace$$

\subsection{Igualdad de Conjuntos}
Dos conjuntos A y B son iguales si $A\subseteq B \land B\subseteq A$. \textbf{Notacion: }$A=B$.
\subsection{Conjuntos Disjuntos}
A y B son disjuntos si $A \cap B=\phi$

\section{Operaciones Binarias}
\begin{description}
\item [Unión: ]$A \cup B=\lbrace x/x \in A \lor x\in B\rbrace$
\item [Intersección: ]$A\cap B=\lbrace x/x \in A \land x\in B\rbrace$
\item [Diferencia: ]$A-B=\lbrace x/x \in A \land x\not\in B\rbrace$
\item [Diferencia Simétrica: ]$A\Delta B=(A-B)\cup(B-A)$
\item [Producto Cartesiano: ]$A\times B=\lbrace(a,b)/a\in A \land b\in B\rbrace$
\end{description}


\section{Leyes de la Teoría de la Computación}
Sea A,B y C conjuntos, se cumplen las siguientes leyes:
\begin{enumerate}
\item \textbf{Idempotencia: }$A\cup A=A \quad A\cap A=A$
\item \textbf{Conmutatividad: }$A\cup B=B\cup A \quad A\cap B=B\cap A$
\item \textbf{Asociatividad: }\\
	$(A\cup B)\cup C=A\cup(B\cup C) \\ (A\cap B)\cap C=A\cap(B\cap C)$
\item \textbf{Distributividad: }\\
	$(A\cup B)\cap C=(A\cap C)\cup(B\cap C) \\ (A\cap B)\cup C=(A\cup C)\cap(B\cup C)$
\item \textbf{Absorción: }\\
	$(A\cup B)\cap A=A \\ (A\cap B)\cup A=A$
\item \textbf{Leyes de Morgan: }\\
	$A-(B\cup C)=(A-B)\cap(A-C) \\ A-(B\cap C)=(A-B)\cup(A-C)$
\end{enumerate}
\textbf{Prueba del item 6: }\\
Sea: $I=A-(B\cup C) \quad , D=(A-B)\cap(A-C)$\\
Se debe cumplir que.$I\subseteq D \land D\subseteq I$
\begin{itemize}
\item PP $I\subseteq D$\\
	Sea $x \in I \rightarrow x\in A \quad pero\; x\not\in(B\cup C)$\\
	$\rightarrow x\in A\; pero\; x\not\in B\land x\not\in C$\\
	$\rightarrow (x\in A\; pero\; x\not\in B)\land(x\in A\; pero\; x\not\in C)$\\
	$x\in(A-B)\land x\in(A-C)$\\
	$x\in(A-B)\cap(A-C)\quad \rightarrow x\in D$
\item PP $D\subseteq I$\\
	Sea $z\in (A-B)\cap(A-C)\\
	z\in(A-B)\land z\in(A-C)\\
	\rightarrow (z\in A\; pero\; z\not\in B)\land(z\in A\; pero\; z\not\in C)\\
	\rightarrow z\in A\; pero\; z\not\in(B\cup C)\\
	\rightarrow z\in(A-(B\cup C))\\
	\rightarrow z\in I\\
	\therefore I=D$
\end{itemize}

\section{Operaciones Generalizadas}
\begin{enumerate}
\item Unión Generalizada:\\
	$\displaystyle\bigcup_{i\in I}A_i=\lbrace x/\exists_i \in I\quad x\in A_i \rbrace$
\item Intersección Generalizada:\\
	$\displaystyle\bigcap_{i\in I}A_i =\lbrace x/x \in A_i \quad \forall_i \in I\rbrace$
\item Producto Cartesiano Generalizado:\\
	$\otimes_{i=1}^n A_i=\lbrace x/x=(x_1,x_2,...,x_n),x_i\in A_i\rbrace$\\
	$A^i=\otimes_{i=1}^n A$
\end{enumerate}

\section{Conjunto Potencia}
Dado un conjunto A, la colección de todos los subconjuntos de A y ella misma se llama conjunto potencia. \textbf{Notación: }$2^A$. \\

\textbf{Ejemplo: }\\
	$A=\lbrace c,d\rbrace\\
	2^A=\lbrace \phi,\lbrace c\rbrace,\lbrace d\rbrace, A\rbrace$
\section{Cardinalidad}
$|A|$ denota la cantidad de elementos del conjunto A.\\
\textbf{Ejemplo: }\\
	$|A|=2\\
	|2^A|=4\quad \quad |2^A|=2^{|A|}$
\section{Partición de un Conjunto}
Sea $A\not=\phi$, una partición de A es un subconjunto $\pi$ de A tal que:
\begin{enumerate}
\item Cada elemento de $\pi$ es no vacío.\\
	$B_i\not=\phi \quad \forall_i \in I$
\item Miembros distintos de $\pi$ deben ser disjuntos.\\
	$B_i \cap B_j =\phi \quad i\not=j$
\item $\bigcup \pi=A \quad\quad \bigcup B_i=A$
\end{enumerate}

\textbf{Ejemplo: }Sea $A=\lbrace a,b,c,d \rbrace$
\begin{itemize}
\item $\lbrace\lbrace a,b\rbrace,\lbrace c\rbrace,\lbrace d\rbrace \rbrace$ (1)si (2)si (3)si, por tanto es una partición.
\item $\lbrace\lbrace b,c\rbrace,\lbrace c,d\rbrace\rbrace$ (1)si (2)no (3)no, por tanto no es una partición.
\end{itemize}

Sea m un entero positivo fijo, se define $\mathds{Z}_i$ como:
$\mathds{Z}_i=\lbrace x/x \in \mathds{Z} \land x-i=k.m, \mbox{para algún entero k} \rbrace$

\textbf{Ejemplo: }Sea m=3 entonces:
\begin{enumerate}
\item Obtener $\mathds{Z}_0,\mathds{Z}_1,\mathds{Z}_2$
\item Verifique si $\mathds{Z}_0,\mathds{Z}_1,\mathds{Z}_2$ es una partición de $\mathds{Z}$.
\end{enumerate}

\textbf{Solución: }
\begin{itemize}
\item $\mathds{Z}_0=\lbrace ...,-6,-3,0,3,6,...\rbrace \\
		\mathds{Z}_1=\lbrace ...,-5,-2,1,4,7,...\rbrace \\
		\mathds{Z}_2=\lbrace ...,-4,-1,2,5,8,...\rbrace$
\item (1)$\mathds{Z}_i\not=\phi$: si cumple.\\
		(2)$\mathds{Z}_0\cap \mathds{Z}_1=\phi, \mathds{Z}_0\cap \mathds{Z}_2=\phi, \mathds{Z}_1\cap\mathds{Z}_2=\phi$: si cumple.\\
		(3)$\bigcup \mathds{Z}_i=\mathds{Z}$: si cumple. Por lo tanto si es una partición.
\end{itemize}

\section{Representación en Computador de Conjuntos}

Supongamos que $\mathfrak{U}$ es finito. Establecemos un orden arbitrario de los elementos de $\mathfrak{U}$. Ejemplo: $a_1,a_2,...,a_n$. Representamos un conjunto A de $\mathfrak{U}$ con la cadena de bits de longitud n, en la cual el i-ésimo bit es ''1'' si $a_i$ pertenece a A, y ''0'' si $a_i\not\in A$.\\
\textbf{Ejemplo: }$\mathfrak{U}=\lbrace 1,2,3,4,5,6,7,8,9,10\rbrace$, supongamos que los elementos están en orden creciente,$a_i=i$. Escriba representaciones en cadena de bits para:
\begin{enumerate}
\item Enteros impares en $\mathfrak{U}$ : \textbf{I}
\item Enteros pares en $\mathfrak{U}$: \textbf{P}
\item Subconjuntos de enteros no mayores que 5.
\end{enumerate}
 
\textbf{Solución: }
\begin{enumerate}
\item La cadena de bits que representa a \textbf{I}:\\
	$1010101010$
\item La cadena de bits que representa a \textbf{P}:\\
	$0101010101$
\item No mayores a 5:\\
	$1111100000$
\end{enumerate}
\textbf{Obs:} 
\begin{itemize}
\item Cadena de bits para el complemento a \textbf{I}: $0101010101$
\item Cadena para la unión de \textbf{I} y \textbf{P}:
\begin{center}
\begin{tabular}{c}
1010101010\\
0101010101\\ \hline
1111111111
\end{tabular}
\end{center}
\end{itemize} 

					


