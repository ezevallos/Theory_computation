\chapter{Lenguajes Formales}

En oposición al lenguaje natural, un lenguaje formal es tal que:
\begin{enumerate}
\item Tiene una sintaxis bien definida. De tal modo que dada una sentencia, es posible saber si pertenece o no al lenguaje.
\item Tiene una semántica precisa. No es posible encontrar sentencias ambiguas o sin significado.
\end{enumerate}
\textbf{Ejemplo: }C, Java, HTML.

\section{Lenguajes Regulares}

Son un tipo de Lenguaje Formal. Reciben este nombre porque sus palabras contienen regularidades o repeticiones de los mismos componentes.

\textbf{Ejemplo: }
\begin{itemize}
\item $L_1=\{cd,cdcd,cdcdcd,...\}$

Las cadenas contienen las subcadenas un número dado de veces.
\item $L_2=\{ abc,cc,abab,abccc,ababc,...\}$

La regularidad consiste encadenas que empiezan con repeticiones de ''ab'' seguidas de repeticiones de c.
\end{itemize}

Se considerará a los lenguajes finitos también regulares.

$L_3=\{ hoy, es, sabado \}$ $L_3$ es regular.

\section{Aplicación}

Se pueden utilizar para especificar la generación de analizadores léxicos.
\subsection{Analizadores Léxicos}

Es un programa que recibe como entrada el código fuente de otro programa y produce como salida Lexemas.

Las palabras están compuestas por lexemas y morfemas.

\subsubsection{Lexema}
Es la raíz o parte de la palabra que no varía.

\textbf{Ejemplo: }\textbf{deport}-e, \textbf{deport}-ivo, \textbf{deport}-ista.
\subsubsection{Morfema}
Es la parte que se le añade al lexema para completar su significado y así generar nuevas palabras.

\textbf{Ejemplo: }moder-\textbf{o}, modern-\textbf{as}, modern-\textbf{ísimo}.

\subsection{Definición Recursiva}

Sea un alfabeto $\Sigma$, Un lenguaje $L\subseteq \Sigma^*$ regular se define recursivamente como sigue.
\begin{itemize}
\item $\phi$, el lenguaje vacío, es un lenguaje regular.
\item $\{ \varepsilon \}$ es un lenguaje regular.
\item $\{ a \}$ es un lenguaje regular $\forall a\in \Sigma$.
\item Si $L_1$ y $L_2$ son lenguaje regulares, entonces $L_1 \cup L_2, L_1 \cdot L_2 ,L_1^*$ son lenguajes regulares.
\item Ningún otro lenguaje sobre $\Sigma$ (definida en las 4 anteriores) es un lenguaje regular.
\end{itemize}

\textbf{Ejemplo: }Sea $\Sigma =\{ a,b \}$. Encuentre 6 lenguajes regulares.

$\begin{array}{rl}
L_1 &=\phi \\
L_2 &=\{a\} \quad L_4\{b\}\\
L_5 &=\{a,b\} \quad L_6=\{ ab\} \quad L_7=\{ a,ab,b\}
\end{array}$

\textbf{Ejemplo: }Dado $\Sigma = \{ a,b\}$. Describa 3 L.R. infinitas y representarlos de forma abreviada.

$L_1=\{ w\in \Sigma^* / w$ empieza con $b\}$ $=\{b\}\Sigma^*$

$L_2=\{ w\in \Sigma^* / w$ contiene exactamente una $a\}$ $=\{b\}^* \cdot\{a\} \cdot\{ b\}^*$

$L_3=\{ w\in \Sigma^* / w \mbox{ contiene a la subcadena } ba\}$ \\
$\begin{array}{p{0.65cm}l}
   &=\{ ba,aaba,bbabba,abbaaab,...\} \\
   &=\Sigma^* \{ ba\}\Sigma^*
\end{array}$

$L_4=\{ w\in\Sigma^* / w$ contiene exactamente dos símbolos $b\}$ $=\{a\}^* \{b\}\{a\}^* \{b\}\{a\}^*$

$L_5=\{a^k /k \geq 0, a\in\Sigma \}$


\section{Características}
Un L.R tiene estas características.
\begin{enumerate}
\item Puede ser escrito mediante una ''Expresión Regular''.
\item Puede ser reconocido mediante un ''Autómata Finito''.
\item Puede ser generado mediante una ''Gramática Regular''.
\end{enumerate}

\section{Expresiones Regulares (ER)}
Una ER es una secuencia de caracteres que forma un patrón de búsqueda. Se usa para especificar un conjunto de cadenas requeridas para un propósito particular.

\textbf{Aplicaciones: }
\begin{itemize}
\item Búsqueda de patrones de cadenas de caracteres.
\item Operaciones de sustitución.
\end{itemize}

\textbf{Ejemplo: }Dadas las cadenas Handel, H\"andel, Haendel. Describa el patrón que los especifique.

H$\{a|\ddot{a}|ae\}$ndel

\subsection{Definición Recursiva}
Dado un alfabeto $\Sigma$, una ER se define recursivamente como sigue:
\begin{itemize}
\item El símbolo $\phi$ es una ER.
\item $\varepsilon$ es una ER.
\item $a\in\Sigma$, es una ER.
\item Si p y q son ER, entonces p.q, p$\cup$q, p* son ER.
\end{itemize}

Se cumple que toda ER sobre $\Sigma$ describe a un LR sobre $\Sigma$.

\textbf{Abreviatura: }

Para los lenguajes siguientes, a continuación se dan las abreviaturas correspondientes.

$\begin{array}{cc}
 & \mbox{Exp. Regulares} \\
\{v\}\cup \{w\} & v+w\\
\{v.w\} & vw\\
\{w\}^* & w^*\\
\{w\}^+ & w^+\\
\end{array}$

\subsection{Nivel de Prioridad}

$\begin{array}{cc}

\mbox{Operador} & \mbox{Prioridad} \\
* & 1 \\
\cdot & 2 \\
+ & 3
\end{array}$

\textbf{Ejemplo: }Reducir la expresión. $E=( \{ a\}^*\{b\} )\cup \{c\}$.

Queda: $a^*b+c$, mediante las ER.

\textbf{Definición: }Dada una expresión regular $\alpha$, denotado por $L(\alpha)$ como:

\begin{itemize}
\item $L(\phi)=\phi$
\item $L(\varepsilon)=\{\varepsilon\}=L_\varepsilon$
\item $L(a)=\{a\}$, donde $a\in\Sigma$
\item $L(\alpha \cdot \beta)= L(\alpha)\cdot L(\beta)$, donde $\alpha, \beta$ son ER.
\item $L(\alpha +\beta)=L(\alpha)\cup L(\beta)$, donde $\alpha, \beta$ son ER.
\item $L(\alpha^*)=(L(\alpha))^*$
\end{itemize}

\section{Equivalencia de ER}

\textbf{Definición: }Dadas las ER $\alpha,\beta$ definidas sobre $\Sigma$, diremos que son equivalentes si ambas generan el mismo lenguaje.

$L(\alpha)=L(\beta)$

\textbf{Notación: }Si $\alpha$ y $\beta$ son equivalentes se denotará $\alpha =\beta$.
\section{Propiedades de las ER}

Sea $\Sigma$ un alfabeto $\alpha,\beta$ y $\gamma$ ER. Se cumple:
\begin{enumerate}
\item $\alpha+\beta=\beta+\alpha$
\item $\alpha+\phi=\alpha=\phi+\alpha$
\item $\alpha+\alpha=\alpha$
\item $(\alpha+\beta)+\gamma=\alpha+(\beta+\gamma)$
\item $\varepsilon \alpha =\alpha=\alpha \varepsilon$
\item $\phi \alpha=\alpha=\alpha\phi$
\item $(\alpha \beta)\gamma= \alpha (\beta \gamma)$ %obs!!
\item $\alpha(\beta+\gamma)=\alpha \beta+\alpha \gamma\\
(\alpha+\beta) \gamma =\alpha\gamma+\beta\gamma$
\item $\phi^* =\varepsilon$
\end{enumerate}

\textbf{Pruebas}

P1:Debemos probar que $\alpha+\beta\subseteq\beta+\alpha; \quad \beta+\alpha\subseteq\alpha+\beta$. 

PPQ $\alpha+\beta\subseteq\beta+\alpha$

Sea $ w\in\alpha+\beta \Rightarrow w\in \alpha \lor w\in\beta$\\
$\begin{array}{p{3cm}l}
					& \Rightarrow w\in\beta \lor w\in\alpha \\ 
					& \Rightarrow w\in \beta +\alpha
					\end{array}$
					
PPQ $\beta+\alpha\subseteq\alpha+\beta$

Sea $z\in \beta+\alpha  \Rightarrow z\in\beta \lor z\in\alpha$\\
$\begin{array}{p{3cm}l}
			& \Rightarrow z\in \alpha \lor z\in \beta \\
			& \Rightarrow z\in\alpha+\beta
\end{array}$

$$\therefore \alpha+\beta=\beta+\alpha$$


P9: Tenemos que $\alpha=\beta$ si $L(\alpha)=L(\beta)$. Bastará probar que $L(\phi^*)=L(\varepsilon)$.

$L(\phi^*)=(L(\phi))^*=\phi^*$

$\begin{array}{rl}
\phi^* = \bigcup_{k=0}^\infty \phi^k &= \phi^0 \cup\phi^1 \cup\phi^2...\\
				&=\phi^0\\
				&=\{\varepsilon\}\\
				&=L(\varepsilon)
				\end{array}$
				
$$\therefore L(\phi^*)=L(\varepsilon)$$
\textbf{OBS: }\\
$\begin{array}{ll}
L(\phi)&=\phi\\
L^* &=\bigcup_{k=0}^\infty L^k \\
L^0 &=\{\varepsilon\}\\
L^1 &=L\\
L^{k+1} &=L^{k}L\\
L(\varepsilon) &=\{\varepsilon\}
\end{array}$

\textbf{Ejemplo: }Sea $\Sigma=\{0,1\}$ y la ER $\alpha=0^*10^*$. Describa a las cadenas de $L(\alpha)$, usando las propiedades.

\textbf{Solución:}

$\begin{array}{rl}
L(\alpha)=L(0^*10^*) &=L(0^*)L(1)L(0^*)\\
		&=(L(0))^* L(1)(L(0))^* \\
		&=\{0\}^*\{1\}\{0\}^*\\
		&=\{0^j.1.0^k / j\geq 0 \quad k\geq 0 \}
\end{array}$